\section*{Exercício 1.3}
\begin{justify}
A Figura 1.14 mostra uma onda triangular. Qual é a frequência fundamental desta onda? Expresse a frequência fundamental em unidades de Hz ou em rad/s.
\begin{figure}[H]
       \centering
       \includegraphics[width=\textwidth,height=\textheight,keepaspectratio]{./Cap1/Figuras/fig114.png}
          \begin{center}
            Figura 1.14 do livro.
        \end{center}
   \end{figure}
   
    \emph{\textbf{Resposta:}} A frequência fundamental, dada por $\omega$ (com unidade em $rad/s$), é dada por
    \begin{align*}
        \omega = \frac{2\pi}{T}
    \end{align*}
    
    Onde $T$ é o período fundamental (dado em $s$). O período fundamental deste sinal é igual a 0,2 segundo. Sendo assim, a frequência fundamental $\omega$ será
    \begin{align*}
        \omega &= \frac{2\pi}{0,2}\\
        \omega &= 10\pi \text{ rad/s}
    \end{align*}
    
    Sabe-se também que a correlação entre a frequência fundamental $\omega$ em rad/s e a frequência fundamental $f$ em Hz é dada por
    \begin{align*}
        \omega = 2\pi f
    \end{align*}
    
    Portanto,
    \begin{align*}
        f &= \frac{\omega}{2\pi} \\
        f &= \frac{10\pi}{2\pi} \\
        f &= 5 \text{ Hz}
    \end{align*}
\end{justify}