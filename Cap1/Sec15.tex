\section*{Exercício 1.15}
\begin{justify}
É possível que um sinal senoidal exponencialmente amortecido de qualquer tipo seja periódico?\\

\emph{\textbf{Resposta: }}Não. Um sinal senoidal exponencialmente amortecido possui a seguinte equação genérica:
    \begin{align*}
        x(t)&=A e^{-\alpha t}\text{ sen}(\omega t+\phi), &\text{com } -\infty<t<\infty\text{,  } \alpha>0 \text{  e  } A\neq 0
    \end{align*}
    
    Ao passo que um sinal senoidal periódico possui a seguinte equação genérica:
    \begin{align*}
        x(t)&=A\text{ sen}(\omega t+\phi), &\text{com } -\infty<t<\infty \text{  e  } A\neq 0
    \end{align*}
    
    A inclusão do termo exponencial amortecido $e^{-\alpha t}$ faz com que o sinal senoidal deixe de ser periódico, uma vez que este passa a ficar cada vez mais atenuado conforme $t\rightarrow\infty$ e cada vez mais pronunciado conforme $t\rightarrow-\infty$.
\end{justify}