\section*{Exercício 1.5}
\begin{justify}
\begin{figure}[H]
       \centering
       \includegraphics[width=\textwidth,height=\textheight,keepaspectratio]{./Cap1/Figuras/fig113.png}
          \begin{center}
            Figura 1.13 do livro.
        \end{center}
\end{figure}

(a) Qual é a energia total do pulso retangular mostrado na Figura 1.13(b)?\\

    \emph{\textbf{Resposta:}} A energia associada ao pulso $x_2(t)$ é dada por
    \begin{align*}
        E &= \int_{-\infty}^{\infty} [x_2(t)]^2 \,dt \\
        E &= \int_{-T_1/2}^{T_1/2} A^2 \,dt \\
        E &= A^2\left[\frac{T_1}{2}-\left(-\frac{T_1}{2}\right)\right] \\
        E &= A^2\left(\frac{2T_1}{2}\right) \\
        E &= A^2T_1
    \end{align*}
\newpage
(b) Qual é a potência média da onda quadrada mostrada na Figura 1.13(a)?\\

    \emph{\textbf{Resposta:}} A potência média da onda $x_1(t)$ é dada por
    \begin{align*}
        P &= \frac{1}{T}\int_{0}^{T} [x_2(t)]^2 \,dt \\
        P &= \frac{1}{T}\left( \int_{0}^{T/2} (1)^2\,dt  + \int_{T/2}^{T} (-1)^2\,dt\right)\\
        P &= \frac{1}{T}\left[ \left( \frac{T}{2} \right) + \left( T - \frac{T}{2} \right)\right]\\
        P &= \frac{1}{T}\left( \frac{T}{2} + \frac{T}{2} \right)\\
        P &= 1
    \end{align*}
\end{justify}