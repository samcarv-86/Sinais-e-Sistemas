\section*{Exercício 1.6}
\begin{justify}
Qual é a potência média da onda triangular mostrada na Figura 1.14?
\begin{figure}[H]
       \centering
       \includegraphics[width=\textwidth,height=\textheight,keepaspectratio]{./Cap1/Figuras/fig114.png}
          \begin{center}
            Figura 1.14 do livro.
            \end{center}
\end{figure}

    \emph{\textbf{Resposta:}}
    A equação da reta que descreve a reta ascendente entre 0 e 0,1 é dada por
    \begin{align*}
        x_1(t) &= m_1(t-t_0)+x_0
    \end{align*}
    
    Onde $m_1$ é o resultado de $\frac{y_f-y_i}{x_f-x_i}$. Calculando $m_1$:
    \begin{align*}
        m_1 &= \frac{1-(-1)}{0,1-0}\\
        m_1 &= \frac{2}{0.1}\\
        m_1 &= 20
    \end{align*}
    
    Sendo $x_0=-1$ e $t_0=0$, tem-se que a expressão que descreve a onda entre 0 e 0,1 é igual a  \begin{align*}
        x_1(t) &= 20t-1
    \end{align*}
    
    A equação da reta que descreve a reta descendente entre 0,1 e 0,2 é dada por
    \begin{align*}
        x_2(t) &= m_2(t-t_1)+x_1
    \end{align*}
    
    Onde $m_2$ é o resultado de $\frac{y_f-y_i}{x_f-x_i}$. Calculando $m_2$:
    \begin{align*}
        m_2 &= \frac{-1-1}{0,2-0,1}\\
        m_2 &= \frac{-2}{0.1}\\
        m_2 &= -20
    \end{align*}
    
     Sendo $x_1=1$ e $t_1=0,1$, tem-se que a expressão que descreve a onda entre 0,1 e 0,2 é igual a  
     \begin{align*}
            x_2(t) &= -20(t+0.1)+1
     \end{align*}
    
     Encontrando agora a potência associada a $x(t)$:
     \begin{align*}
            P &= \frac{1}{T}\int_{0}^{T} [x(t)]^2 \,dt\\
            P &= \frac{1}{T}\left(\int_{0}^{T/2} [x_1(t)]^2\,dt + \int_{T/2}^{T} [x_2(t)]^2\,dt\right)\\
            P &= \frac{1}{T}\left(\int_{0}^{T/2} [20t-1]^2\,dt + \int_{T/2}^{T} [-20(t+0.1)+1]^2\,dt\right)\\
            P &= \frac{1}{T}\left(\int_{0}^{T/2} [400t^2-40t+1]\,dt + \int_{T/2}^{T} [-20t-2]^2\,dt\right)\\
            P &= \frac{1}{T}\left(\int_{0}^{T/2} [400t^2-40t+1]\,dt + \int_{T/2}^{T} [400t^2-120t+9]\,dt\right)\\
            P &= \frac{1}{T}\left(\left[\frac{400t^3}{3}-\frac{40t^2}{2}+t\right]_0^{T/2} + \left[\frac{400t^3}{3}-\frac{120t^2}{2}+9t\right]_{T/2}^T\right)\\
            P &= \frac{1}{T}\left( \frac{50T^3}{3}-5T^2+\frac{T}{2}+\frac{1}{30}  \right)
     \end{align*}
     
     E, com $T=0,2$ se chega a
     \begin{align*}
            P &= \frac{1}{3}
     \end{align*}     
\end{justify}


