\section*{Exercício 1.11}
\begin{justify}
O sinal de tempo discreto $x[n]$ é definido por
\begin{equation*}
    x[n]=
    \Bigg\{
    \begin{NiceMatrix}[l]
        1, & n=1 \text{ e } n=2 \\ -1, & n=-1 \text{ e } n=-2 \\ 0, & n=0 \text{ e } |n|>2
    \end{NiceMatrix}
\end{equation*}
Encontre o sinal $y[n]=x[n+3]$ deslocado no tempo.\\

    \emph{\textbf{Resposta:}} O novo sinal deslocado no tempo é assim obtido:
        \begin{align*}
            x[n+3]=
                \Bigg\{
                \begin{NiceMatrix}[l]
                    1, & n+3=1 \text{ e } n+3=2 \\ -1, & n+3=-1 \text{ e } n+3=-2 \\ 0, & n+3=0 \text{ e } |n+3|>2
                \end{NiceMatrix}
        \end{align*}
        
        Sabe-se que $|n+3|>2$ pode ser escrito como sendo duas condições, ou seja, 
        \begin{align*}
            x[n+3]=
                \Bigg\{
                \begin{NiceMatrix}[l]
                    1, & n+3=1 \text{ e } n+3=2 \\ -1, & n+3=-1 \text{ e } n+3=-2 \\ 0, & n+3=0 \text{, } n+3>2 \text{ e } -(n+3)>2
                \end{NiceMatrix}
        \end{align*}
        
        Reorganizando os termos:
        \begin{align*}
            x[n+3]=
                \Bigg\{
                \begin{NiceMatrix}[l]
                    1, & n=-2 \text{ e } n=-1 \\ -1, & n=-4 \text{ e } n=-5 \\ 0, & n=-3 \text{, } n>-1 \text{ e } n<-5
                \end{NiceMatrix}
        \end{align*}
\end{justify}