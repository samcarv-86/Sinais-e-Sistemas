\section*{Exercício 1.4}
\begin{justify}
Qual é a frequência fundamental da onda quadrada de tempo discreto mostrada na Figura 1.15?
\begin{figure}[H]
       \centering
       \includegraphics[width=\textwidth,height=\textheight,keepaspectratio]{./Cap1/Figuras/fig115.png}
          \begin{center}
            Figura 1.15 do livro.
        \end{center}
\end{figure}

    \emph{\textbf{Resposta:}} Se observa que o sinal possui 4 pulsos positivos seguidos de 4 pulsos negativos, e passa a repetir-se neste padrão. Portanto o período de $x[n]$ vale $N=8$ amostras.\\
    
    Desta maneira, a frequência fundamental $\Omega$ do sinal discreto $x[n]$ será
    \begin{align*}
        \Omega &= \frac{2\pi}{N}\\
        \Omega &= \frac{2\pi}{8}\\
        \Omega &= \frac{\pi}{4} \text{ radianos}
\end{align*}
\end{justify}


