\section*{Exercício 1.9}
\begin{justify}
O sinal de tempo discreto $x[n]$ é definido por
\begin{equation*}
    x[n]=
    \Bigg\{
    \begin{NiceMatrix}[l]
    1, & n=1 \\ -1, & n=-1 \\ 0, & n=0 \text{ e } |n|>1
    \end{NiceMatrix}
\end{equation*}
Encontre o sinal composto $y[n]$ definido em termos de $x[n]$ por
\begin{equation*}
    y[n]=x[n]+x[-n]
\end{equation*}

    \emph{\textbf{Resposta:}} A parcela $x[-n]$ é assim atingida:
        \begin{align*}
            x[-n]=
            \Bigg\{
            \begin{NiceMatrix}[l]
            1, & -n=1 \\ -1, & -n=-1 \\ 0, & -n=0 \text{ e } |-n|>1
            \end{NiceMatrix}
        \end{align*}
        
        Ou seja,
        \begin{align*}
            x[-n]=
            \Bigg\{
            \begin{NiceMatrix}[l]
            1, & n=-1 \\ -1, & n=1 \\ 0, & n=0 \text{ e } |n|>1
            \end{NiceMatrix}
        \end{align*}
        
        A soma das parcelas $x[n]$ e $x[-n]$ resulta em $y[n]=0$ para qualquer valor de n.
\end{justify}