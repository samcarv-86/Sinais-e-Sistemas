\section*{Exercício 1.13}
\begin{justify}
Considere os seguintes sinais senoidais, determinando se cada $x[n]$ é periódico. Se for, encontre seu período fundamental.\\

(a) $x[n]=5$ sen$(2n)$

(b) $x[n]=3$ cos$(0,2\pi n)$

(c) $x[n]=5$ cos$(6\pi n)$

(d) $x[n]=5$ sen$(6\pi n/35)$\\

    \emph{\textbf{Resposta: }}Para que um sinal discreto seja periódico é necessário que seu período fundamental $N$ \textbf{seja um número inteiro} e que \textbf{haja um número inteiro} $m$ que satisfaça a igualdade $N=\frac{2\pi m}{\Omega}$, sendo $\Omega$ a frequência fundamental do sinal em questão.\\
    
    Na questão (a) se observa que a frequência fundamental do sinal é $\Omega=2$ rad. Com isto,
    \begin{align*}
        N&=\frac{2\pi m}{2}\\
        N&=\pi m
    \end{align*}
    Ou seja, se verifica que não existe um número inteiro $m$ que torne o período fundamental $N$ um número inteiro. Desta forma, o sinal $x[n]$ da questão (a) \textbf{não é periódico}.\\
 
    Na questão (b) se observa que a frequência fundamental do sinal é $\Omega=0,2\pi$ rad. Com isto,
    \begin{align*}
        N&=\frac{2\pi m}{0,2\pi}\\
        N&=10 m
    \end{align*}
    
    Ou seja, se verifica que existe pelo menos um número inteiro $m$ que torna o período fundamental $N$ um número inteiro. Desta forma, o sinal $x[n]$ da questão (b) \textbf{é periódico}, e seu período fundamental é aquele cujo menor valor de $m$ satisfaz a condição de que $N$ seja inteiro, ou seja, $N=20$ e $m=1$.\\

    Na questão (c) se observa que a frequência fundamental do sinal é $\Omega=6\pi$ rad. Com isto,
    \begin{align*}
        N&=\frac{2\pi m}{6\pi}\\
        N&=\frac{m}{3}
    \end{align*}
    
    Ou seja, se verifica que existe pelo menos um número inteiro $m$ que torna o período fundamental $N$ um número inteiro. Desta forma, o sinal $x[n]$ da questão (c) \textbf{é periódico}, e seu período fundamental é aquele cujo menor valor de $m$ satisfaz a condição de que $N$ seja inteiro, ou seja, $N=1$ e $m=3$.\\

    Na questão (d) se observa que a frequência fundamental do sinal é $\Omega=6\pi/35$ rad. Com isto,
    \begin{align*}
        N&=\frac{2\pi m}{6\pi/35}\\
        N&=\frac{35m}{3}
    \end{align*}
    
    Ou seja, se verifica que existe pelo menos um número inteiro $m$ que torna o período fundamental $N$ um número inteiro. Desta forma, o sinal $x[n]$ da questão (d) \textbf{é periódico}, e seu período fundamental é aquele cujo menor valor de $m$ satisfaz a condição de que $N$ seja inteiro, ou seja, $N=35$ e $m=3$.\\
\end{justify}