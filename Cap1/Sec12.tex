\section*{Exercício 1.12}
\begin{justify}
Considere um sinal de tempo discreto $x[n]$ definido por
\begin{equation*}
    x[n]=
    \Big\{
    \begin{NiceMatrix}[l]
        1, & -2\le n \le 2 \\ 0, & |n|>2
    \end{NiceMatrix}
\end{equation*}
Encontre $y[n]=x[3n-2]$.\\

    \emph{\textbf{Resposta: }}O novo sinal é assim obtido:
    \begin{align*}
        x[3n-2]=
        \Big\{
        \begin{NiceMatrix}[l]
            1, & -2\le (3n-2) \le 2 \\ 0, & |3n-2|>2
        \end{NiceMatrix}
    \end{align*}
    
    Sabe-se que $|3n-2|>2$ pode ser escrito como sendo duas condições, ou seja,
    \begin{align*}
        x[3n-2]=
        \Big\{
        \begin{NiceMatrix}[l]
            1, & -2\le (3n-2) \le 2 \\ 0, & (3n-2)>2 \text{ e } -(3n-2)>2
        \end{NiceMatrix}
    \end{align*}
    
    Sabe-se também que $-2\le (3n-2) \le 2$ também pode ser escrito na forma de duas condições, ou seja,
    \begin{align*}
        x[3n-2]=
        \Big\{
        \begin{NiceMatrix}[l]
            1, & 3n-2 \le 2 \text{ e } 3n-2 \ge -2\\ 
            0, & 3n-2>2 \text{ e } -(3n-2)>2
        \end{NiceMatrix}
    \end{align*}
    
    Reorganizando os termos:
    \begin{align*}
        x[3n-2]=
        \Big\{
        \begin{NiceMatrix}[l]
            1, & n \le \sfrac{4}{3} \text{ e } n \ge 0\\ 
            0, & n>\sfrac{4}{3} \text{ e } n<0
        \end{NiceMatrix}
    \end{align*}
    
    Como os instantes de tempo $-\sfrac{4}{3}$ e $\sfrac{4}{3}$ não existem no domínio discreto (afinal toda amostra ocorre em um instante que, \textbf{por definição}, é igual a um número inteiro), tais amostras são perdidas. Assim, o sinal $x[3n-2]$ fica sendo
    \begin{align*}
        x[3n-2]=
        \Big\{
        \begin{NiceMatrix}[l]
            1, & n=0 \text{ e } n=1\\ 
            0, & \text{caso contrário}
        \end{NiceMatrix}
    \end{align*}
\end{justify}