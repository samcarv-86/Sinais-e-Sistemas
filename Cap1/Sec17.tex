\section*{Exercício 1.17}
\begin{justify}
Expresse o operador $H$ que descreve a relação entrada-saída
    \begin{align*}
        y[n]&=\frac{1}{3}\left( x[n+1]+x[n]+x[n-1]\right)
    \end{align*}
em termos do operador $S$ de deslocamento de tempo.\\

\emph{\textbf{Resposta: }}A saída $y[n]$ de um sistema que realiza a operação $H$ sobre uma entrada $x[n]$ pode ser expressa por
    \begin{align*}
        y[n]&=H\{x[n]\}
    \end{align*}

O operador $S^k$ desloca o tempo de um dado sinal na forma $x[n-k]$. Por exemplo, aplicar o operador $S^{1}$ a um sinal $x[n]$ significa que este sinal foi atrasado de 1 unidade, passando a ser representado como $x[n-1]$.\\

Desta maneira o operador $H$ que descreve a operação proposta no exercício é expresso como
    \begin{align*}
        H&=\frac{1}{3}\left( S^{-1} + S^0 + S^1 \right)\\
        H&=\frac{1}{3}\left( S^{-1} + 1 + S^1 \right)
    \end{align*}
\end{justify}